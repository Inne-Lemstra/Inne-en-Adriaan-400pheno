%de abstract
\documentclass[12pt,a4paper, twocolumn]{article}
\voffset = -70pt
\textheight = 662pt
\author{Adriaan van der Graaf, Inne Lemstra}
\usepackage{sweave}
\title{400 Pheno}
\date{Maart 2012}
\begin{document}
\maketitle

\subsubsection*{Introductie}
\textit{Arabidopsis thaliana} (zandraket) is een kleine snelgroeiende plant die veel gebruikt wordt in het biologisch onderzoek.
Van de arabidopsis plant zijn 165 indivduen (zaden) voortgekomen uit een RIL (recomninant inbred line) van twee subsoorten van arabidopsis Bayreuth x shahdara.
Van deze 165 individuen zijn 5 eigenschappen gemeten onder een verschijdene omstandigheden
Ook is van 69 markers  gemeten van welke ouder elk individu de marker had gekregen (Resp. AA en BB).
Genetisch gelijke individuen zijn gebruikt om dit experiment in drie oogsten (batches) uit te voeren, twee tegelijkertijd en een op een later tijdstip.
De 5 klassieke eigenschappen die gemeten zijn:
Gmax is de maximum ontkieming,
U8416 is de tijd tussen 16\% en 84\% van  ontkieming,
T10 is de tijd totdat 10\% van alle zaden ontkiemt is,
T50 is de tijd voordat 50\% van alle zaden ontkiemt is,
AUC is de oppervlakte onder de ontkiemings grafiek  tot 100 uur na het experiment.
Deze parameters werden gemeten op 8 verschillende milieus namenlijk:
Manitol (een suiker alcohol), salt (NaCl), ABA (een hormoon), hete en koude omgevingen.
Al deze omgevingen en eigenschappen zijn ook met en zonder koude stratificatie gedaan. 
Wat het totaal uiteindelijk op 404 gedefinieerde 'traits" brengt\\
\subsubsection*{Methode}
Het bestand dat we hebben gebruikt is de BayShatraitsAll. 
Een ';' gesepareerde tabel met de eerste 404 kolommen phenotypische data en de laatste 69 kolommen de genotypische data.
Waar de eerste twee rijen van phenotypen niks bevatten en de genotypen de chromosomen en de morgans.
Er werd een t-test gemaakt per trait en per marker (matrix van phenotypes bij genotypes).
Deze matrix werd uitgezet over de markers en er werd een peakfinding algoritme op losgelaten. (plaatje
De pieken die boven de logarithm of odds (LOD) 3 zitten (P waarden maal -log 10 , dus 3 is 0,001 kans dat de trait in de normale verdeling zou zitten) //
Peak finder Grafiek//
Daar wordt een matrix van gemaakt op de eerste kolom een phenotype en op de tweede een genotype met de -10log P-waarden en de  verhoudingswaarden AA/BB en AA-BB.
deze zelfde methode wordt gebruikt voor een anova per genotype. 
Hiervoor zijn verschillende functies geschreven om een zo duidelijk mogelijk code te schrijven(voor zover dat mogelijk is met nieuwe programmeurs).
Daarnaast ook om functies die vaker gebruikt worden niet opnieuw te hoeven bedenken.

Het maken van de multiple Anova had wat meer voeten in de aarde omdat we graag de verschillen tussen de batches en milieus eruit wilden halen.
Hiervoor is een linear model opgesteld met waarbij eerst het verschillende batch-effect eruit werd gefilterd en daarna de milieu-effecten.
Zo krijgen we een afgewogen oordeel over de verschillende eigenschappen.
de verschillende eigenschappen,batches en milieus zijn in de volgende tabel te vinden.\\

\begin{tabular} {l  l  l}
\multicolumn{3}{c}{Phenotypische metingen} \\
Eigenschappen & Batches & Milieus\\ 
\hline 
Gmax & A & AfterRipening\\
U8416 & B & NaCL\\
T10 & C & Mannitol\\
T50 & D & Cold\\
AUC & ABC & Heat\\
 &  & ABA\\
 &  & Fresh\\
 &  & Stratification\\
\end{tabular}\\
\subsubsection*{Resultaten}
Na het uitwerken van de multiple anova en het afkappen op een LOD van 3 of hoger zijn er op 3 verschillende manieren markers bekend. 
Op de t.test manier, de Anova per trait manier en de multiple anova manier. 
Deze matrices geven aan op welke trait ze slaan en welke marker significant is voor de specifieke trait.
Omdat  op dat moment alleen nog LOD waarden voor handen waren, was het ook interesant te weten welke waarden groter zijn dan de andere. 
Dit is gedaan door de gemiddelde waarden BB (per trait) af te trekken van de waarden AA (AA-BB).
Als deze waarde positief is, dan zal de AA variant van de marker een groter effect hebben op de trait. 
Als de AA-BB negatief is, dan zal de BB variant een groter effect hebben op de trait. \\
Met deze waarden is te bepalen of een specifiek organisme een bepaalde marker van de vader of de moeder nodig heeft.
Als er wordt gekozen voor de maximale opbrengst dan zou de volgende grafiek een mooi uitgangspunt zijn.\\
GRAFIEK\\
Daarnaast is het ook mogelijk om over het chromosoom te bekijken wat de belangrijkste regio's zijn.\\
Zo kun je zien op welk deel van het chromosoom de meeste Quantative Trait Loci zitten.
\subsection*{Conclusies en interpretatie}
Op dit moment kan er worden gekeken 


\end{document}